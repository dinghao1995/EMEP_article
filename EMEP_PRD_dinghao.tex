%\documentclass[14pt,hyperref={CJKbookmarks=true}]{beamer}
%\usetheme{AnnArbor}
%\setbeamercolor{normal test}{bg=black!10}
\documentclass[a4paper]{article}
\usepackage{hyperref}
%%%%%%%%%%%%%%%%%%%%%%%%%%%%%%%%%%%%%%%%%%%%%%%%%%%%%%%%%%%%%%%%%%%%%%%%%%%
\begin{document}
\title{EMEP/MSC-W模型的珠三角本地化及其模拟性能评估}
\author[常鸣]{常鸣 \and 丁浩 \and 马明睿 \and 王雪梅*}
\institute[ECI]{暨南大学环境与气候研究院}
% \date{Apr 12th, 2018}
% \maketitle
% \begin{frame}
% \titlepage
% \end{frame}
%%%%%%%%%%%%%%%%%%%%%%%%%%%%%%%%%%%%%%%%%%%%%%%%%%%%%%%%%%%%%%%%%%%%%%%%%%%
\begin[abstract]{摘要}
% 最后写
\end[abstract]{摘要}
\tableofcontents
%%%%%%%%%%%%%%%%%%%%%%%%%%%%%%%%%%%%%%%%%%%%%%%%%%%%%%%%%%%%%%%%%%%%%%%%%%%
%%%%%%%%%%%%%%%%%%%%%%%%%%%%%%%%%%%%%%%%%%%%%%%%%%%%%%%%%%%%%%%%%%%%%%%%%%%
\section{引言}
% \begin{frame}{引言}
% \begin{EMEP/MSC-W模型简介}
EMEP(European Monitoring and Evaluation Programme)隶属于远距离跨界空气污染公约Convention on Long-range Transboundary Air Pollution (CLRTAP),旨在解决跨界空气污染问题。该公约旗下包含5个EMEP中心,其中一个为坐落于挪威奥斯陆的MSC-W。
EMEP/MSC-W模型是由挪威气象局开发的化学传输模型(自1977年至今),最初EMEP/MSC-W模型以50KM*50KM的分辨率覆盖整个欧洲,垂直方向上可从地面延伸到
对流层顶(100hPa)
 
% 丁浩、明睿撰写,常鸣修改
% \end{EMEP/MSC-W模型简介}
% \end{frame}{引言}
%%%%%%%%%%%%%%%%%%%%%%%%%%%%%%%%%%%%%%%%%%%%%%%%%%%%%%%%%%%%%%%%%%%%%%%%%%%
%%%%%%%%%%%%%%%%%%%%%%%%%%%%%%%%%%%%%%%%%%%%%%%%%%%%%%%%%%%%%%%%%%%%%%%%%%%
\section{模式设置}
% 常鸣撰写
\subsection{模式机制设置}
\subsection{气象驱动场}
本研究采用WRF气象模式提供精细化的EMEP模式气象驱动场。WRF模式由FNL全球在分析数据驱动,其网格设置如图\ref{网格设置}所示。

\subsection{数据来源}

%%%%%%%%%%%%%%%%%%%%%%%%%%%%%%%%%%%%%%%%%%%%%%%%%%%%%%%%%%%%%%%%%%%%%%%%%%%
%%%%%%%%%%%%%%%%%%%%%%%%%%%%%%%%%%%%%%%%%%%%%%%%%%%%%%%%%%%%%%%%%%%%%%%%%%%
\section{模式输入本地化}
\subsection{排放源清单}
% 明睿整理撰写
\subsection{土地利用资料}
% 丁浩整理撰写
\subsection{其他本地化输入资料}
% 常鸣整理撰写
%%%%%%%%%%%%%%%%%%%%%%%%%%%%%%%%%%%%%%%%%%%%%%%%%%%%%%%%%%%%%%%%%%%%%%%%%%%
%%%%%%%%%%%%%%%%%%%%%%%%%%%%%%%%%%%%%%%%%%%%%%%%%%%%%%%%%%%%%%%%%%%%%%%%%%%
\section{结果与讨论}
\subsection{模拟结果验证}
\subsubsection{气象要素模拟结果验证}
% 丁浩整理撰写
\subsubsection{污染物模拟结果验证}
% 丁浩整理撰写
%%%%%%%%%%%%%%%%%%%%%%%%%%%%%%%%%%%%%%%%%%%%%%%%%%%%%%%%%%%%%%%%%%%%%%%%%%%
\subsection{土地利用资料本地化的影响}
% 下周进一步实验
\subsubsection{边界层结构差异}
\subsubsection{污染物时空分布差异}
%%%%%%%%%%%%%%%%%%%%%%%%%%%%%%%%%%%%%%%%%%%%%%%%%%%%%%%%%%%%%%%%%%%%%%%%%%%
\subsection{排放源清单本地化的影响}
% 下周进一步实验
\subsubsection{污染物时空分布差异}
% \subsubsection{未来减排情景效果评估}
% 这部分拆到另一篇明睿来主导写。
%%%%%%%%%%%%%%%%%%%%%%%%%%%%%%%%%%%%%%%%%%%%%%%%%%%%%%%%%%%%%%%%%%%%%%%%%%%
%\subsection{作为预报组件的模拟效率评估}
% 常鸣整理撰写
\section{结论与不足}
%%%%%%%%%%%%%%%%%%%%%%%%%%%%%%%%%%%%%%%%%%%%%%%%%%%%%%%%%%%%%%%%%%%%%%%%%%%
%%%%%%%%%%%%%%%%%%%%%%%%%%%%%%%%%%%%%%%%%%%%%%%%%%%%%%%%%%%%%%%%%%%%%%%%%%%
\end{document}
