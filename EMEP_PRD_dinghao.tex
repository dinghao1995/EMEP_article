%\documentclass[14pt,hyperref={CJKbookmarks=true}]{beamer}
%\usetheme{AnnArbor}
%\setbeamercolor{normal test}{bg=black!10}
\documentclass[a4paper]{article}
\usepackage{hyperref}
%%%%%%%%%%%%%%%%%%%%%%%%%%%%%%%%%%%%%%%%%%%%%%%%%%%%%%%%%%%%%%%%%%%%%%%%%%%
\begin{document}
\title{EMEP/MSC-W模型的珠三角本地化及其模拟性能评估}
\author[常鸣]{常鸣 \and 丁浩 \and 马明睿 \and 王雪梅*}
\institute[ECI]{暨南大学环境与气候研究院}
% \date{Apr 12th, 2018}
% \maketitle
% \begin{frame}
% \titlepage
% \end{frame}
%%%%%%%%%%%%%%%%%%%%%%%%%%%%%%%%%%%%%%%%%%%%%%%%%%%%%%%%%%%%%%%%%%%%%%%%%%%
\begin[abstract]{摘要}
% 最后写
\end[abstract]{摘要}
\tableofcontents
%%%%%%%%%%%%%%%%%%%%%%%%%%%%%%%%%%%%%%%%%%%%%%%%%%%%%%%%%%%%%%%%%%%%%%%%%%%
%%%%%%%%%%%%%%%%%%%%%%%%%%%%%%%%%%%%%%%%%%%%%%%%%%%%%%%%%%%%%%%%%%%%%%%%%%%
\section{引言}
% \begin{frame}{引言}
% \begin{EMEP/MSC-W模型简介}
EMEP(European Monitoring and Evaluation Programme)隶属于远距离跨界空气污染公约Convention on Long-range Transboundary Air Pollution (CLRTAP),旨在解决跨界空气污染问题。该公约旗下包含5个EMEP中心,其中一个为坐落于挪威奥斯陆的MSC-W。
EMEP/MSC-W模型是由挪威气象局开发的化学传输模型(自1977年至今),最初EMEP/MSC-W模型以50KM*50KM的分辨率覆盖整个欧洲,垂直方向上可从地面延伸到
对流层顶(100hPa)
% 丁浩、明睿撰写,常鸣修改
% \end{EMEP/MSC-W模型简介}
% \end{frame}{引言}
%%%%%%%%%%%%%%%%%%%%%%%%%%%%%%%%%%%%%%%%%%%%%%%%%%%%%%%%%%%%%%%%%%%%%%%%%%%
%%%%%%%%%%%%%%%%%%%%%%%%%%%%%%%%%%%%%%%%%%%%%%%%%%%%%%%%%%%%%%%%%%%%%%%%%%%
\section{模式设置}
% 常鸣撰写
\subsection{WRF气象驱动设置}

本研究采用WRF气象模式提供精细化的EMEP模式气象驱动场。WRF模式由FNL全球在分析数据驱动,其网格设置如图\ref{wangge}所示。模拟采用四重Domain,中心点设置在珠三角(23.5°N,113.7°E),采用LAMBERT投影方式。模拟试验采用的物理参数化方案如表\ref{physicsschemes}所示。

\begin{figure}[!htb]
\centering\includegraphics[height=\dimexpr\pagegoal-\pagetotal-4\baselineskip\relax,
width=\textwidth,keepaspectratio]{figwangge.jpg} 
\caption{网格设置}\label{wangge}
\end{figure} 

\begin{threeparttable}[!htb]\small%\footnotesize
	\centering
	\caption{WRF物理参数化方案}\label{physicsschemes}
	\begin{tabular}{ccc}
		\toprule 
		物理过程 & 方案名称 & 参考文献 \\
		\midrule
		云微物理方案 & Lin Scheme & \citep{lin1983bulk} \\
		长波辐射方案 & RRTMG Longwave & \citep{iacono2008radiative} \\
		短波辐射方案 & Goddard Shortwave & \citep{chou2001thermal} \\
		近地面层方案 & Monin-Obukhov (Janjic Eta) & \citep{janjic1994step} \\
		陆面过程方案 & Noah LSM & \citep{mukul2004implementation} \\
		行星边界层方案        & Mellor-Yamada-Janjic PBL & \citep{janjic1994step} \\
		积云参数化方案        & Kain-Fritsch & \citep{kain1993convective} \\
		城市冠层方案          & Muilt-Layer BEP & \citep{salamanca2010new} \\
		\bottomrule
	\end{tabular}
\end{threeparttable}

\subsection{EMEP模式机制设置}
%EMEP的设置为2个case,其中case:simWithoutCB是由WRF-d04结果直接驱动EMEP模拟,case:simWithCB是由WRF-d01驱动EMEP生产出化学边界(ChemBoundary)之后nested到d04的EMEP模型结果。

\subsection{数据来源}

%%%%%%%%%%%%%%%%%%%%%%%%%%%%%%%%%%%%%%%%%%%%%%%%%%%%%%%%%%%%%%%%%%%%%%%%%%%
%%%%%%%%%%%%%%%%%%%%%%%%%%%%%%%%%%%%%%%%%%%%%%%%%%%%%%%%%%%%%%%%%%%%%%%%%%%
\section{模式输入本地化}
\subsection{排放源清单}
% 明睿整理撰写
\subsection{土地利用资料}
% 丁浩整理撰写
\subsection{其他本地化输入资料}
% 常鸣整理撰写
%%%%%%%%%%%%%%%%%%%%%%%%%%%%%%%%%%%%%%%%%%%%%%%%%%%%%%%%%%%%%%%%%%%%%%%%%%%
%%%%%%%%%%%%%%%%%%%%%%%%%%%%%%%%%%%%%%%%%%%%%%%%%%%%%%%%%%%%%%%%%%%%%%%%%%%
\section{结果与讨论}
\subsection{模拟结果验证}
\subsubsection{气象要素模拟结果验证}
% 丁浩整理撰写
\subsubsection{污染物模拟结果验证}
% 丁浩整理撰写
%%%%%%%%%%%%%%%%%%%%%%%%%%%%%%%%%%%%%%%%%%%%%%%%%%%%%%%%%%%%%%%%%%%%%%%%%%%
\subsection{土地利用资料本地化的影响}
% 下周进一步实验
\subsubsection{边界层结构差异}
\subsubsection{污染物时空分布差异}
%%%%%%%%%%%%%%%%%%%%%%%%%%%%%%%%%%%%%%%%%%%%%%%%%%%%%%%%%%%%%%%%%%%%%%%%%%%
\subsection{排放源清单本地化的影响}
% 下周进一步实验
\subsubsection{污染物时空分布差异}
% \subsubsection{未来减排情景效果评估}
% 这部分拆到另一篇明睿来主导写。
%%%%%%%%%%%%%%%%%%%%%%%%%%%%%%%%%%%%%%%%%%%%%%%%%%%%%%%%%%%%%%%%%%%%%%%%%%%
%\subsection{作为预报组件的模拟效率评估}
% 常鸣整理撰写
\section{结论与不足}
%%%%%%%%%%%%%%%%%%%%%%%%%%%%%%%%%%%%%%%%%%%%%%%%%%%%%%%%%%%%%%%%%%%%%%%%%%%
%%%%%%%%%%%%%%%%%%%%%%%%%%%%%%%%%%%%%%%%%%%%%%%%%%%%%%%%%%%%%%%%%%%%%%%%%%%
\end{document}
